%\documentclass[10pt,twocolumn, a4paper]{article}
\documentclass[a4paper,twocolumn, 10pt, landscape]{article}
\usepackage[brazil]{babel}
\usepackage[utf8]{inputenc}

\usepackage[cm]{fullpage}

\usepackage[left=1cm,top=1cm,right=1cm,bottom=1cm,nohead,nofoot]{geometry}


\usepackage{listings} %% Para codigos fonte
\usepackage{float}

\lstset{numbers=left,
stepnumber=1,
firstnumber=1,
numberstyle=\tiny,
extendedchars=true,
breaklines=true,
frame=tb,
captionpos=b,
tabsize=3, % sets default tabsize to 2 spaces
basicstyle=\footnotesize,
stringstyle=\ttfamily,
showstringspaces=false
}
\renewcommand{\lstlistingname}{Código}
\renewcommand{\lstlistlistingname}{Lista de Listagens}



\begin{document}
\listoftables
\lstlistoflistings

\section{Tabelas}

\begin{table}[!h]
% use packages: array
\begin{tabular}{llll}
tipo & bits & min...max & precisao \\ 
char & 8 & 0..127 & 2 \\ 
signed char & 8 & -128..127 & 2 \\ 
unsigned char & 8 & 0..255 & 2 \\ 
short & 16 & -32.768 .. 32.767 & 4 \\ 
unsigned short & 16 & 0 .. 65.535 & 4 \\ 
int & 32 & -2x10**9 .. 2 x 10**9 & 9 \\ 
unsigned int & 32 & 0 .. 4x10**9 & 9 \\ 
int64\_t & 64 & -9 x 10**18 .. 9 x 10**18 & 18 \\ 
uint64\_t & 64 & 0 .. 18 x 10**18 & 19
\end{tabular}
\caption{Limites de representação de dados}
\label{limites}
\end{table}


\begin{table}
% use packages: array
\begin{tabular}{ll}
0! = 1 \\
1! = 1 \\
2! = 2 \\
3! = 6 \\
4! = 24 \\
5! = 120 \\
6! = 720 \\
7! = 5.040 \\
8! = 40.320 \\
9! = 362.880 \\
10! = 3.628.800 \\
11! = 39.916.800 \\
12! = 479.001.600 [limite do (unsigned) int] \\
13! = 6.227.020.800 \\
14! = 87.178.291.200 \\
15! = 1.307.674.368.000 \\
16! = 20.922.789.888.000 \\
17! = 355.687.428.096.000 \\
18! = 6.402.373.705.728.000 \\
19! = 121.645.100.408.832.000 \\
20! = 2.432.902.008.176.640.000 [limite do (u)int64\_t] \\
\end{tabular}
\caption{Fatorial}
\label{Fatorial}
\end{table}

\begin{table}
% use packages: array
\begin{tabular}{ll}
Tipe & \% \\ 
char & c \\ 
int & d \\ 
float & e, E, f, g, G \\ 
int (octal) & o \\ 
int (hexa) & x, X \\ 
uint & u \\ 
char* & s
\end{tabular}
\caption{scanf() - \%[*][width][modifiers]type}
\label{tipos}
\end{table}

\begin{table}
 \begin{tabular}{ll}
modifiers & tipo \\ 
h & short int (d, i, n), or unsigned short int (o, u, x) \\ 
l & long int (d, i, n), or unsigned long int (o, u, x), or double (e, f, g) \\ 
L & long double (e, f, g) \\ 
\end{tabular}
\caption{scanf() \%[*][width][modifiers]type}
\label{modifiers}
\end{table}

\section{Codigos}

\lstinputlisting[language=c, label=modelo, caption={Modelo}]{/home/vod/ederfm/maratona/libmaratonadcc/doc/src/modelo.cpp}
\lstinputlisting[language=c, label=cmp, caption={comparcao de ponto flutuante}]{/home/vod/ederfm/maratona/libmaratonadcc/doc/src/cmp.cpp}
\lstinputlisting[language=c, label=vim, caption={.vimrc para a configuração do vim}]{/home/vod/ederfm/maratona/libmaratonadcc/doc/src/vim.cpp}
\lstinputlisting[language=c, label=printf, caption={printf}]{/home/vod/ederfm/maratona/libmaratonadcc/doc/src/printf.cpp}
\lstinputlisting[language=c, label=pq, caption={exemplo de priority queue}]{/home/vod/ederfm/maratona/libmaratonadcc/doc/src/priority_queue_ex.cpp}
\lstinputlisting[language=c, label=multset, caption={exemplo de multset}]{/home/vod/ederfm/maratona/libmaratonadcc/doc/src/multset_ex.cpp}

\end{document}
