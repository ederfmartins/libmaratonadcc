%\documentclass[10pt,twocolumn, a4paper]{article}
\documentclass[a4paper,twocolumn, 10pt, landscape]{article}
\usepackage[brazil]{babel}
\usepackage[utf8]{inputenc}

\usepackage[cm]{fullpage}

\usepackage[left=1cm,top=1cm,right=1cm,bottom=1cm,nohead,nofoot]{geometry}


\usepackage{listings} %% Para codigos fonte
\usepackage{float}

\lstset{numbers=left,
stepnumber=1,
firstnumber=1,
numberstyle=\tiny,
extendedchars=true,
breaklines=true,
frame=tb,
captionpos=t,
tabsize=3, % sets default tabsize to 2 spaces
basicstyle=\footnotesize,
stringstyle=\ttfamily,
showstringspaces=false,
numbersep=5pt, % how far the line-numbers are from the code
%frame=single, % adds a frame around the code
xleftmargin=17pt,
framexleftmargin=17pt,
framexrightmargin=0pt,
framexbottommargin=4pt
}
\renewcommand{\lstlistingname}{Código}
\renewcommand{\lstlistlistingname}{Algoritmos}



\begin{document}
\tableofcontents
\listoftables
\lstlistoflistings

\newpage

\section{Tabelas}

\begin{table}[!h]
% use packages: array
\begin{tabular}{llll}
tipo & bits & min...max & precisao \\ 
char & 8 & 0..127 & 2 \\ 
signed char & 8 & -128..127 & 2 \\ 
unsigned char & 8 & 0..255 & 2 \\ 
short & 16 & -32.768 .. 32.767 & 4 \\ 
unsigned short & 16 & 0 .. 65.535 & 4 \\ 
int & 32 & -2x10**9 .. 2 x 10**9 & 9 \\ 
unsigned int & 32 & 0 .. 4x10**9 & 9 \\ 
int64\_t & 64 & -9 x 10**18 .. 9 x 10**18 & 18 \\ 
uint64\_t & 64 & 0 .. 18 x 10**18 & 19
\end{tabular}
\caption{Limites de representação de dados}
\label{limites}
\end{table}


\begin{table}[!h]
% use packages: array
\begin{tabular}{ll}
0! = 1 \\
1! = 1 \\
2! = 2 \\
3! = 6 \\
4! = 24 \\
5! = 120 \\
6! = 720 \\
7! = 5.040 \\
8! = 40.320 \\
9! = 362.880 \\
10! = 3.628.800 \\
11! = 39.916.800 \\
12! = 479.001.600 [limite do (unsigned) int] \\
13! = 6.227.020.800 \\
14! = 87.178.291.200 \\
15! = 1.307.674.368.000 \\
16! = 20.922.789.888.000 \\
17! = 355.687.428.096.000 \\
18! = 6.402.373.705.728.000 \\
19! = 121.645.100.408.832.000 \\
20! = 2.432.902.008.176.640.000 [limite do (u)int64\_t] \\
\end{tabular}
\caption{Fatorial}
\label{Fatorial}
\end{table}

\begin{table}
% use packages: array
\begin{tabular}{ll}
Tipo & \% \\ 
char & c \\ 
int & d \\ 
float & e, E, f, g, G \\ 
int (octal) & o \\ 
int (hexa) & x, X \\ 
uint & u \\ 
char* & s
\end{tabular}
\caption{scanf() - \%[*][width][modifiers]type}
\label{tipos}
\end{table}

\begin{table}
 \begin{tabular}{ll}
modifiers & tipo \\ 
h & short int (d, i, n), or unsigned short int (o, u, x) \\ 
l & long int (d, i, n), or unsigned long int (o, u, x), or double (e, f, g) \\ 
ll & long long int (d, i, n), or unsigned long long int (o, u, x) \\ 
L & long double (e, f, g) \\ 
\end{tabular}
\caption{scanf() \%[*][width][modifiers]type}
\label{modifiers}
\end{table}


\begin{table}
 \begin{tabular}{lll}
função & descrição \\
atof & Convert string to double \\
atoi & Convert string to integer \\
atol & Convert string to long integer \\
strtod & Convert string to double \\
strtol & Convert string to long integer \\
strtoul & Convert string to unsigned long integer \\
\end{tabular}
\caption{stdlib}
\label{stdlib functions}
\end{table}

\begin{table}
 \begin{tabular}{lll}
função & descrição \\
cos &	Compute cosine  \\
sin &	Compute sine  \\
tan &	Compute tangent  \\
acos &	Compute arc cosine  \\
asin &	Compute arc sine  \\
atan &	Compute arc tangent  \\
atan2 &	Compute arc tangent with two parameters  \\
cosh &	Compute hyperbolic cosine  \\
sinh &	Compute hyperbolic sine  \\
tanh &	Compute hyperbolic tangent  \\
exp &	Compute exponential function  \\
frexp &	Get significand and exponent  \\
ldexp &	Generate number from significand and exponent  \\
log &	Compute natural logarithm  \\
log10 &	Compute common logarithm  \\
modf &	Break into fractional and integral parts  \\
pow &	Raise to power  \\
sqrt &	Compute square root  \\
ceil &	Round up value  \\
fabs &	Compute absolute value  \\
floor &	Round down value  \\
fmod &	Compute remainder of division  \\
\end{tabular}
\caption{math (angulos em radianos)}
\label{math functions}
\end{table}

%\begin{table}
\begin{tabular}{llllllllllll}
2 & 3 & 5 & 7 & 11 & 13 & 17 & 19 & 23 & 29 & 31 \\ 
37 & 41 & 43 & 47 & 53 & 59 & 61 & 67 & 71 & 73 & 79 \\ 
83 & 89 & 97 & 101 & 103 & 107 & 109 & 113 & 127 & 131 & 137 \\ 
139 & 149 & 151 & 157 & 163 & 167 & 173 & 179 & 181 & 191 & 193 \\ 
197 & 199 & 211 & 223 & 227 & 229 & 233 & 239 & 241 & 251 & 257 \\ 
263 & 269 & 271 & 277 & 281 & 283 & 293 & 307 & 311 & 313 & 317 \\ 
331 & 337 & 347 & 349 & 353 & 359 & 367 & 373 & 379 & 383 & 389 \\ 
397 & 401 & 409 & 419 & 421 & 431 & 433 & 439 & 443 & 449 & 457 \\ 
461 & 463 & 467 & 479 & 487 & 491 & 499 & 503 & 509 & 521 & 523 \\ 
541 & 547 & 557 & 563 & 569 & 571 & 577 & 587 & 593 & 599 & 601 \\ 
607 & 613 & 617 & 619 & 631 & 641 & 643 & 647 & 653 & 659 & 661 \\ 
673 & 677 & 683 & 691 & 701 & 709 & 719 & 727 & 733 & 739 & 743 \\ 
751 & 757 & 761 & 769 & 773 & 787 & 797 & 809 & 811 & 821 & 823 \\ 
827 & 829 & 839 & 853 & 857 & 859 & 863 & 877 & 881 & 883 & 887 \\ 
907 & 911 & 919 & 929 & 937 & 941 & 947 & 953 & 967 & 971 & 977 \\ 
983 & 991 & 997 & 1009 & 1013 & 1019 & 1021 & 1031 & 1033 & 1039 & 1049 \\ 
1051 & 1061 & 1063 & 1069 & 1087 & 1091 & 1093 & 1097 & 1103 & 1109 & 1117 \\ 
1123 & 1129 & 1151 & 1153 & 1163 & 1171 & 1181 & 1187 & 1193 & 1201 & 1213 \\ 
1217 & 1223 & 1229 & 1231 & 1237 & 1249 & 1259 & 1277 & 1279 & 1283 & 1289 \\ 
1291 & 1297 & 1301 & 1303 & 1307 & 1319 & 1321 & 1327 & 1361 & 1367 & 1373 \\ 
1381 & 1399 & 1409 & 1423 & 1427 & 1429 & 1433 & 1439 & 1447 & 1451 & 1453 \\ 
1459 & 1471 & 1481 & 1483 & 1487 & 1489 & 1493 & 1499 & 1511 & 1523 & 1531 \\ 
1543 & 1549 & 1553 & 1559 & 1567 & 1571 & 1579 & 1583 & 1597 & 1601 & 1607 \\ 
1609 & 1613 & 1619 & 1621 & 1627 & 1637 & 1657 & 1663 & 1667 & 1669 & 1693 \\ 
1697 & 1699 & 1709 & 1721 & 1723 & 1733 & 1741 & 1747 & 1753 & 1759 & 1777 \\ 
1783 & 1787 & 1789 & 1801 & 1811 & 1823 & 1831 & 1847 & 1861 & 1867 & 1871 \\ 
1873 & 1877 & 1879 & 1889 & 1901 & 1907 & 1913 & 1931 & 1933 & 1949 & 1951 \\ 
1973 & 1979 & 1987 & 1993 & 1997 & 1999 & 2003 & 2011 & 2017 & 2027 & 2029 \\ 
2039 & 2053 & 2063 & 2069 & 2081 & 2083 & 2087 & 2089 & 2099 & 2111 & 2113 \\ 
2129 & 2131 & 2137 & 2141 & 2143 & 2153 & 2161 & 2179 & 2203 & 2207 & 2213 \\ 
2221 & 2237 & 2239 & 2243 & 2251 & 2267 & 2269 & 2273 & 2281 & 2287 & 2293 \\ 
2297 & 2309 & 2311 & 2333 & 2339 & 2341 & 2347 & 2351 & 2357 & 2371 & 2377 \\ 
2381 & 2383 & 2389 & 2393 & 2399 & 2411 & 2417 & 2423 & 2437 & 2441 & 2447 \\ 
2459 & 2467 & 2473 & 2477 & 2503 & 2521 & 2531 & 2539 & 2543 & 2549 & 2551 \\ 
2557 & 2579 & 2591 & 2593 & 2609 & 2617 & 2621 & 2633 & 2647 & 2657 & 2659 \\ 
2663 & 2671 & 2677 & 2683 & 2687 & 2689 & 2693 & 2699 & 2707 & 2711 & 2713 \\ 
2719 & 2729 & 2731 & 2741 & 2749 & 2753 & 2767 & 2777 & 2789 & 2791 & 2797 \\ 
2801 & 2803 & 2819 & 2833 & 2837 & 2843 & 2851 & 2857 & 2861 & 2879 & 2887 \\ 
2897 & 2903 & 2909 & 2917 & 2927 & 2939 & 2953 & 2957 & 2963 & 2969 & 2971 \\ 
2999 & 3001 & 3011 & 3019 & 3023 & 3037 & 3041 & 3049 & 3061 & 3067 & 3079 \\ 
3083 & 3089 & 3109 & 3119 & 3121 & 3137 & 3163 & 3167 & 3169 & 3181 & 3187 \\ 
3191 & 3203 & 3209 & 3217 & 3221 & 3229 & 3251 & 3253 & 3257 & 3259 & 3271 \\ 
3299 & 3301 & 3307 & 3313 & 3319 & 3323 & 3329 & 3331 & 3343 & 3347 & 3359 \\ 
3361 & 3371 & 3373 & 3389 & 3391 & 3407 & 3413 & 3433 & 3449 & 3457 & 3461 \\ 
3463 & 3467 & 3469 & 3491 & 3499 & 3511 & 3517 & 3527 & 3529 & 3533 & 3539 \\ 
3541 & 3547 & 3557 & 3559 & 3571 & 3581 & 3583 & 3593 & 3607 & 3613 & 3617 \\ 
3623 & 3631 & 3637 & 3643 & 3659 & 3671 & 3673 & 3677 & 3691 & 3697 & 3701 \\ 
3709 & 3719 & 3727 & 3733 & 3739 & 3761 & 3767 & 3769 & 3779 & 3793 & 3797 \\ 
3803 & 3821 & 3823 & 3833 & 3847 & 3851 & 3853 & 3863 & 3877 & 3881 & 3889 \\ 
3907 & 3911 & 3917 & 3919 & 3923 & 3929 & 3931 & 3943 & 3947 & 3967 & 3989 \\ 
4001 & 4003 & 4007 & 4013 & 4019 & 4021 & 4027 & 4049 & 4051 & 4057 & 4073 \\ 
4079 & 4091 & 4093 & 4099 & 4111 & 4127 & 4129 & 4133 & 4139 & 4153 & 4157 \\ 
4159 & 4177 & 4201 & 4211 & 4217 & 4219 & 4229 & 4231 & 4241 & 4243 & 4253 \\ 
4259 & 4261 & 4271 & 4273 & 4283 & 4289 & 4297 & 4327 & 4337 & 4339 & 4349 \\ 
4357 & 4363 & 4373 & 4391 & 4397 & 4409 & 4421 & 4423 & 4441 & 4447 & 4451 \\ 
4457 & 4463 & 4481 & 4483 & 4493 & 4507 & 4513 & 4517 & 4519 & 4523 & 4547 \\ 
4549 & 4561 & 4567 & 4583 & 4591 & 4597 & 4603 & 4621 & 4637 & 4639 & 4643 \\ 
4649 & 4651 & 4657 & 4663 & 4673 & 4679 & 4691 & 4703 & 4721 & 4723 & 4729 \\ 
4733 & 4751 & 4759 & 4783 & 4787 & 4789 & 4793 & 4799 & 4801 & 4813 & 4817 \\ 
4831 & 4861 & 4871 & 4877 & 4889 & 4903 & 4909 & 4919 & 4931 & 4933 & 4937 \\ 
4943 & 4951 & 4957 & 4967 & 4969 & 4973 & 4987 & 4993 & 4999 & 5003 & 5009 \\ 
5011 & 5021 & 5023 & 5039 & 5051 & 5059 & 5077 & 5081 & 5087 & 5099 & 5101 \\ 
5107 & 5113 & 5119 & 5147 & 5153 & 5167 & 5171 & 5179 & 5189 & 5197 & 5209 \\ 
5227 & 5231 & 5233 & 5237 & 5261 & 5273 & 5279 & 5281 & 5297 & 5303 & 5309 \\ 
5323 & 5333 & 5347 & 5351 & 5381 & 5387 & 5393 & 5399 & 5407 & 5413 & 5417 \\ 
5419 & 5431 & 5437 & 5441 & 5443 & 5449 & 5471 & 5477 & 5479 & 5483 & 5501 \\ 
5503 & 5507 & 5519 & 5521 & 5527 & 5531 & 5557 & 5563 & 5569 & 5573 & 5581 \\ 
5591 & 5623 & 5639 & 5641 & 5647 & 5651 & 5653 & 5657 & 5659 & 5669 & 5683 \\ 
5689 & 5693 & 5701 & 5711 & 5717 & 5737 & 5741 & 5743 & 5749 & 5779 & 5783 \\ 
5791 & 5801 & 5807 & 5813 & 5821 & 5827 & 5839 & 5843 & 5849 & 5851 & 5857 \\ 
5861 & 5867 & 5869 & 5879 & 5881 & 5897 & 5903 & 5923 & 5927 & 5939 & 5953 \\ 
5981 & 5987 & 6007 & 6011 & 6029 & 6037 & 6043 & 6047 & 6053 & 6067 & 6073 \\ 
6079 & 6089 & 6091 & 6101 & 6113 & 6121 & 6131 & 6133 & 6143 & 6151 & 6163 \\ 
6173 & 6197 & 6199 & 6203 & 6211 & 6217 & 6221 & 6229 & 6247 & 6257 & 6263 \\ 
6269 & 6271 & 6277 & 6287 & 6299 & 6301 & 6311 & 6317 & 6323 & 6329 & 6337 \\ 
6343 & 6353 & 6359 & 6361 & 6367 & 6373 & 6379 & 6389 & 6397 & 6421 & 6427 \\ 
6449 & 6451 & 6469 & 6473 & 6481 & 6491 & 6521 & 6529 & 6547 & 6551 & 6553 \\ 
6563 & 6569 & 6571 & 6577 & 6581 & 6599 & 6607 & 6619 & 6637 & 6653 & 6659 \\ 
6661 & 6673 & 6679 & 6689 & 6691 & 6701 & 6703 & 6709 & 6719 & 6733 & 6737 \\ 
6761 & 6763 & 6779 & 6781 & 6791 & 6793 & 6803 & 6823 & 6827 & 6829 & 6833 \\ 
6841 & 6857 & 6863 & 6869 & 6871 & 6883 & 6899 & 6907 & 6911 & 6917 & 6947 \\ 
6949 & 6959 & 6961 & 6967 & 6971 & 6977 & 6983 & 6991 & 6997 & 7001 & 7013 \\ 
7019 & 7027 & 7039 & 7043 & 7057 & 7069 & 7079 & 7103 & 7109 & 7121 & 7127 \\ 
7129 & 7151 & 7159 & 7177 & 7187 & 7193 & 7207 & 7211 & 7213 & 7219 & 7229 \\ 
7237 & 7243 & 7247 & 7253 & 7283 & 7297 & 7307 & 7309 & 7321 & 7331 & 7333 \\ 
7349 & 7351 & 7369 & 7393 & 7411 & 7417 & 7433 & 7451 & 7457 & 7459 & 7477 \\ 
7481 & 7487 & 7489 & 7499 & 7507 & 7517 & 7523 & 7529 & 7537 & 7541 & 7547 \\ 
7549 & 7559 & 7561 & 7573 & 7577 & 7583 & 7589 & 7591 & 7603 & 7607 & 7621 \\ 
7639 & 7643 & 7649 & 7669 & 7673 & 7681 & 7687 & 7691 & 7699 & 7703 & 7717 \\ 
7723 & 7727 & 7741 & 7753 & 7757 & 7759 & 7789 & 7793 & 7817 & 7823 & 7829 \\ 
7841 & 7853 & 7867 & 7873 & 7877 & 7879 & 7883 & 7901 & 7907 & 7919 & 7927 \\ 
7933 & 7937 & 7949 & 7951 & 7963 & 7993 & 8009 & 8011 & 8017 & 8039 & 8053 \\ 
8059 & 8069 & 8081 & 8087 & 8089 & 8093 & 8101 & 8111 & 8117 & 8123 & 8147 \\ 
8161 & 8167 & 8171 & 8179 & 8191 & 8209 & 8219 & 8221 & 8231 & 8233 & 8237 \\ 
8243 & 8263 & 8269 & 8273 & 8287 & 8291 & 8293 & 8297 & 8311 & 8317 & 8329 \\ 
8353 & 8363 & 8369 & 8377 & 8387 & 8389 & 8419 & 8423 & 8429 & 8431 & 8443 \\
8447 & 8461 & 8467 & 8501 & 8513 & 8521 & 8527 & 8537 & 8539 & 8543 & 8563 \\ 
8573 & 8581 & 8597 & 8599 & 8609 & 8623 & 8627 & 8629 & 8641 & 8647 & 8663 \\ 
8669 & 8677 & 8681 & 8689 & 8693 & 8699 & 8707 & 8713 & 8719 & 8731 & 8737 \\ 
8741 & 8747 & 8753 & 8761 & 8779 & 8783 & 8803 & 8807 & 8819 & 8821 & 8831 \\ 
8837 & 8839 & 8849 & 8861 & 8863 & 8867 & 8887 & 8893 & 8923 & 8929 & 8933 \\ 
8941 & 8951 & 8963 & 8969 & 8971 & 8999 & 9001 & 9007 & 9011 & 9013 & 9029 \\ 
9041 & 9043 & 9049 & 9059 & 9067 & 9091 & 9103 & 9109 & 9127 & 9133 & 9137 \\ 
9151 & 9157 & 9161 & 9173 & 9181 & 9187 & 9199 & 9203 & 9209 & 9221 & 9227 \\ 
9239 & 9241 & 9257 & 9277 & 9281 & 9283 & 9293 & 9311 & 9319 & 9323 & 9337 \\ 
9341 & 9343 & 9349 & 9371 & 9377 & 9391 & 9397 & 9403 & 9413 & 9419 & 9421 \\ 
9431 & 9433 & 9437 & 9439 & 9461 & 9463 & 9467 & 9473 & 9479 & 9491 & 9497 \\ 
9511 & 9521 & 9533 & 9539 & 9547 & 9551 & 9587 & 9601 & 9613 & 9619 & 9623 \\ 
9629 & 9631 & 9643 & 9649 & 9661 & 9677 & 9679 & 9689 & 9697 & 9719 & 9721 \\ 
9733 & 9739 & 9743 & 9749 & 9767 & 9769 & 9781 & 9787 & 9791 & 9803 & 9811 \\ 
9817 & 9829 & 9833 & 9839 & 9851 & 9857 & 9859 & 9871 & 9883 & 9887 & 9901 \\ 
9907 & 9923 & 9929 & 9931 & 9941 & 9949 & 9967 & 9973 & & & \\
\end{tabular}
\caption{tabela de primos}
\label{pimesTable}
\end{table}


\newpage

\section{Codigos}

\subsection{Exemplos}
\lstinputlisting[language=c, label=modelo, caption={Modelo}]{src/modelo.cpp}
\lstinputlisting[language=c, label=cmp, caption={comparcao de ponto flutuante}]{src/cmp.cpp}
\lstinputlisting[language=c, label=vim, caption={.vimrc para a configuração do vim}]{src/vim.cpp}
\lstinputlisting[language=c, label=cin, caption={função que acelara o cin. Não deve ser usada com printf}]{src/cinFast.cpp}
\lstinputlisting[language=c, label=printf, caption={printf}]{src/printf.cpp}
\lstinputlisting[language=c, label=map, caption={exemplo de map}]{src/map_ex.cpp}
\lstinputlisting[language=c, label=multset, caption={exemplo de set e multset}]{src/multset_ex.cpp}
\lstinputlisting[language=c, label=list, caption={exemplo de list}]{src/list_ex.cpp}
\lstinputlisting[language=c, label=queue, caption={exemplo de queue}]{src/queue_ex.cpp}
\lstinputlisting[language=c, label=pq, caption={exemplo de priority queue}]{src/priority_queue_ex.cpp}
\lstinputlisting[language=c, label=stack, caption={exemplo de stack}]{src/stack_ex.cpp}
\lstinputlisting[language=c, label=vector, caption={exemplo de vector}]{src/vector_ex.cpp}
\lstinputlisting[language=c, label=string, caption={exemplo de string}]{src/ex_string.cpp}
\lstinputlisting[language=c, label=stringstream, caption={exemplo de stringstream}]{src/sstream_ex.cpp}

\lstinputlisting[language=c, label=sort, caption={exemplo de ordenação}]{src/sort_ex.cpp}
\lstinputlisting[language=c, label=bsearch, caption={pesquisa binária}]{src/bsearch_ex.cpp}

\lstinputlisting[language=c, label=arredondamento, caption={Arredondamento e output em outras bases}]{src/arredondamento_ex.cpp}

\subsection{Teoria dos números}
\lstinputlisting[language=c, label=mdc, caption={máximo divisor comum e mínimo multiplo comum}]{src/mdc_mmc.cpp}
\lstinputlisting[language=c, label=primo, caption={decide se um número é primo}]{src/isPrime.cpp}
\lstinputlisting[language=c, label=fatores, caption={Retorna a fatoração em números primos de abs(n).}]{src/primeMap.cpp}
\lstinputlisting[language=c, label=mpow, caption={Calcula Valor de $a^b mod$ n de forma rápida.}]{src/mpow.cpp}
\lstinputlisting[language=c, label=mulmod, caption={Calcula (a*b)\%c de forma rápida.}]{src/mulmod.cpp}
\lstinputlisting[language=c, label=axbmodc, caption={ Computa x tal que a*x = b (mod c). Quando a equação não tem solução, retorna algum valor arbitrário errado, mas basta conferir o resultado.}]{src/axbmodc.cpp}
\lstinputlisting[language=c, label=discreteLogarithm, caption={\textbf{Baby-step Giant-step algorithm} Calcula o menor valor de e para $b^e = n$ mod p. Retorna -1 se eh impossivel}]{src/discreteLogarithm.cpp}

\subsection{Estruturas de dados}
\lstinputlisting[language=c, label=bigint, caption={Números de precisão harbitrária.}]{src/bigint.cpp}

\subsection{Programação Dinâmica}
\lstinputlisting[language=c, label=subsetsum, caption={\textbf{Sub Set Sum}: Verifica se há um sobconjunto dos elementos do vetor cuja soma seja igual a soma pedida.}]{src/subSetSum.cpp}
\lstinputlisting[language=c, label=lis1d, caption={\textbf{Lis: longest increasing (decreasing) subsequence} O($n^2$)}]{src/lis1d.cpp}
\lstinputlisting[language=c, label=lis1dfast, caption={\textbf{Lis: longest increasing subsequence} O(n*logn)}]{src/lis1dfast.cpp}

\subsection{Grafos}
\lstinputlisting[language=c, label=aciclico, caption={Verifica se o grafo é aciclico.}]{src/aciclico.cpp}
\lstinputlisting[language=c, label=dijkstra, caption={\textbf{Dijkstra} Caminho minimo 1 para todos pesos positivos.}]{src/dijkstra.cpp}
\lstinputlisting[language=c, label=setForest, caption={Floresta dijunta de arvores}]{src/dijointSetForest.cpp}
\lstinputlisting[language=c, label=kruskal, caption={\textbf{Kruskal} Arvore geradora mínima kruskal}]{src/kruskal.cpp}
\lstinputlisting[language=c, label=bipartido, caption={verifica se um grafo é bipartido}]{src/bipartido.cpp}
\lstinputlisting[language=c, label=topsort, caption={faz a ordenação topológica de um grafo acíclico}]{src/topSort.cpp}
\lstinputlisting[language=c, label=maxflow, caption={calcula fluxo máximo, \textbf{Ford-Fulkerson}}]{src/maxFlow.cpp}
\lstinputlisting[language=c, label=maxflowFast, caption={calcula fluxo máximo, algoritmo mais eficiente porém muito maior em tempo de codificação}]{src/maxFlowFast.cpp}

\subsection{Geometria}
\lstinputlisting[language=c, label=ponto, caption={ponto e poligono}]{src/ponto.cpp}
\lstinputlisting[language=c, label=between, caption={Decide se q está sobre o segmento fechado pr.}]{src/Between.cpp}
\lstinputlisting[language=c, label=emComum, caption={Decide se os segmentos fechados pq e rs têm pontos em comum. }]{src/pontosEmComum.cpp}
\lstinputlisting[language=c, label=segdist, caption={Calcula a distância do ponto r ao segmento pq. }]{src/segDist.cpp}
\lstinputlisting[language=c, label=inpoly, caption={Classifica o ponto p em relação ao polígono T. Retorna 0, -1 ou 1 dependendo se p está no exterior, na fronteira ou no interior de T, respectivamente.}]{src/inpoly.cpp}

\subsection{Algebra Linear}
\lstinputlisting[language=c, label=simplex, caption={Simplex}]{src/simplex.cpp}

\subsection{Casamento de strings}
\lstinputlisting[language=c, label=kmp, caption={String matching - Algoritmo \textbf{KMP} - O( n + m)}]{src/kmp.cpp}

\subsection{Outros}
\lstinputlisting[language=c, label=josephus, caption={josephus problem}]{src/josephus.cpp}
\lstinputlisting[language=c, label=nextPermutation, caption={Gera as permutações dos elementos da string}]{src/nextPermutation.cpp}
\lstinputlisting[language=c, label=backtrakingEx, caption={Exemplo de geração de permutações dos elementos da string, usando backtracking}]{src/backtrakingEx.cpp}

% TODO: grafos ( componentes conexas, pontos de articulação e pontes), sistemas lineares, simplex, polinomios.

% std::ios::sync_with_stdio(false);
\section{Biblioteca C/C++}
\subsection{I/O}
Ignorando entradas na família scanf:
\begin{lstlisting}[language=c, label=cio, caption={Ignora os dois floats do meio. Retornará 2 no sucesso.}]
 scanf("%f %*f %*f %d", &a, &b);
\end{lstlisting}


\subsection{Map}
\lstinputlisting[language=c++, label=map, caption={Referencias map}]{src/map.cpp}


\end{document}
